\section{Nombre del proyecto.}
\textbf{“AliveEarth”}

\section{Justificación o propósito.}
El propósito de “AliveEarth” es agilizar el tiempo de análisis de 
la calidad del suelo mediante la técnica de Cromatografía en papel.

\section{Objetivo.}
Ayudar en la mejora del productor agropecuario, ya que es importante 
conocer las condiciones del suelo con el fin de tomar acciones que mejore 
las características para una mayor producción.

\section{Descripción.}
El producto de este trabajo está dirigido principalmente al sector agropecuario. 
Se espera que los resultados y recomendaciones de este sistema permita mejorar la 
calidad de los suelos a través de la implementación de alternativas sustentables que 
incrementarán su productividad.

\subsection{Datos.}
\begin{center}
    \begin{tabular}{|p{5cm}|p{7cm}|}
        \hline
        Empresa/Organización & MayaSoft\\
        \hline
        Proyecto & “AliveEarth”\\
        \hline
        Fecha de inicio & Junio 06, 2019\\
        \hline
        Patrocinador principal & Ecología, Sustentabilidad e Innovación \\
        \hline
        Gerente de proyecto & Milton Neftalí Hernández Jiménez\\
        \hline
    \end{tabular}
\end{center}
\subsection{Patrocinador(es).}
\begin{center}
    \begin{tabular}{|p{5cm}|p{3cm}|p{5cm}|}
        \hline
        \textbf{Nombre} & \textbf{Cargo} & \textbf{Departamento/División} \\
        \hline
        UPCH & & Ingeniería en Desarrollo de Software \\
        \hline
        UPCH & & Ingeniería en Tecnología Ambiental \\
        \hline
    \end{tabular}
\end{center}

\section{Razón de cierre.}
\begin{center}
    \begin{tabular}{|p{10cm}|p{3cm}|}
        \hline
        Entrega  de todos los productos  de conformidad  con los requerimientos  del cliente. & \\
        \hline
        Entrega  parcial de productos  y  cancelación  de otros de conformidad  con los requerimientos  del cliente. & \\
        \hline
        Cancelación de todos los productos asociados con el proyecto. & \\
        \hline
    \end{tabular}
\end{center}

\section{Aceptación de los productos o entregables.}
\begin{center}
    A continuación  se establece cuales  entregables  de proyecto  han  sido aceptados:
    \begin{tabular}{|p{4cm}|p{3cm}|p{6cm}|}
        \hline
        \rowcolor{lightgray} \textbf{Entregable} & \textbf{Aceptación(Si o No)} & \textbf{Observaciones} \\
        \hline
        Aplicación Móvil & & \\
        \hline
        Aplicación Web & & \\
        \hline
        AI & & \\
        \hline
        Validación por correo& & \\
        \hline
        Servidor & & \\
        \hline
        Manual de programador & & \\
        \hline
        Manual de usuario & & \\
        \hline
    \end{tabular}
\end{center}

\section{Aprobaciones.}
\begin{center}
    \begin{tabular}{|p{4cm}|p{3cm}|p{6cm}|}
        \hline
        & & \\
        \hline
        & & \\
        \hline
        & & \\
        \hline
        & & \\
        \hline
    \end{tabular}
\end{center}