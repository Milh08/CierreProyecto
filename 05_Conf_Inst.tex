\section{Configuración e Instalación.}
Para poder modificar, ejecutar el proyecto hecho en Android studio, se debe tomar en cuenta los requerimientos del sistema que Andriod stuido necesita para el buen manejo del entorno. (ver la página oficial de Android studio en la sección de System Requirements https://developer.android.com/studio)

\subsection{Instalación del IDE de Android studio.}
\begin{itemize}
    \item Descargar el IDE de Android studio.
    \item Seguir todos los pasos que el instalador hace mención. (dentro de los pasos de instalación hace referencia a agregar la ruta del ejecutable en el PATH del sistema, es necesario aceptar esa opción)
    \item Ejecutar el IDE de Android studio y dejar instalar todos los componentes que automáticamente se ejecutan.
\end{itemize}

\subsection{Configuración del IDE de Android studio con el uso del proyecto.}
\begin{itemize}
\item Descargar el proyecto del repositorio de GitHub \\ https://github.com/Milhernandez08/ProyetoAppCroma.git.
    \item Abrir el proyecto en el IDE de Android studio.
    \item Descargar todos los componentes y módulos que sean requeridos para poder modificar, ejecutar el proyecto.
\end{itemize}

\subsection{Configuración del dispositivo móvil con sistema operativo Android.}
\begin{itemize}
    \item Ir a ajustes del dispositivo.
    \item En la sección de Sistema, buscar el apartado de Acerca del teléfono.
    \item En la sección de Versión de Software presionar las veces necesarias hasta que salga la notificación de que el equipo esta para uso del desarrollador.
\end{itemize}